\documentclass[12pt]{article}

\usepackage[a4paper,margin=2cm]{geometry}

\usepackage{amsmath}
\usepackage{amssymb}
\usepackage{mathtools}

\usepackage{listings}

\usepackage{booktabs} % For tables
\usepackage[table,xcdraw]{xcolor} % For tables

\usepackage{enumerate}
\usepackage{enumitem}

\usepackage{nameref}

\usepackage{xcolor}

\definecolor{codegreen}{rgb}{0,0.6,0}
\definecolor{codegray}{rgb}{0.5,0.5,0.5}
\definecolor{codepurple}{rgb}{0.58,0,0.82}
\definecolor{backcolour}{rgb}{0.95,0.95,0.92}

\lstdefinestyle{mystyle}{
    backgroundcolor=\color{backcolour},
    commentstyle=\color{codegreen},
    keywordstyle=\color{magenta},
    numberstyle=\tiny\color{codegray},
    stringstyle=\color{codepurple},
    basicstyle=\ttfamily\footnotesize,
    breakatwhitespace=false,
    breaklines=true,
    captionpos=b,
    keepspaces=true,
    numbers=left,
    numbersep=5pt,
    showspaces=false,
    showstringspaces=false,
    showtabs=false,
    tabsize=2
}

\lstset{style=mystyle}

\DeclarePairedDelimiter\abs{\lvert}{\rvert}
\DeclarePairedDelimiter\Abs{\lVert}{\rVert}

\usepackage{fancyhdr}

\pagestyle{fancy}
\lhead{\today}
\chead{Exercise 03\\Algorithmic Foundations of Data Science}
\rhead{Fabian Grob\\Simon Michau\\Til Mohr}

\setlength{\headheight}{50pt}

\begin{document}

\section*{Exercise 1}
\begin{enumerate}[label= (\alph*)]
	\item	Using Theorem 3.6: With \(|\mathcal{H}| = 3^3 = 27\)
			\begin{align*}
				\Pr_{T ~ \mathcal{D}^m} &\left( \forall h \in \mathcal{H}: \abs{err_T(h) - err_D(h)} \leq \epsilon \right) > 1 - \delta \\
				\Pr_{T ~ \mathcal{D}^m} &\left( \forall h \in \mathcal{H}: \abs{err_T(h) - err_D(h)} \leq \epsilon \right) > 0.9 \\
				\Rightarrow &\delta = 0.1 \\\\
				m &\geq \frac{1}{2\epsilon^2} \log \left( \frac{2\abs{\mathcal{H}}}{\delta} \right) \\
				143 &\geq \frac{1}{2\epsilon^2} \log \left( \frac{2 \cdot 3^3}{0.1} \right) \\
				143 &\geq \frac{1}{2\epsilon^2} (\log(54) - \log(0.1)) \\
				143 &\geq \frac{1}{2\epsilon^2} (\log(54) - \log(0.1)) \\
				\epsilon^2 &\geq \frac{(\log(54) - \log(0.1))}{143 \dot 2} \\
				\abs{\epsilon} &\geq \sqrt{\frac{(\log(54) - \log(0.1))}{286}} \\
				\Rightarrow &\epsilon \geq \sqrt{\frac{(\log(54) - \log(0.1))}{286}} \\
				\Pr_{T ~ \mathcal{D}^m} &\left( \forall h \in \mathcal{H}: \abs{err_T(h) - err_D(h)} \leq \epsilon \right) > 0.9 \\
				\Pr_{T ~ \mathcal{D}^m} &\left( \forall h \in \mathcal{H}: \abs{0.03 - err_D(h)} \leq \sqrt{\frac{(\log(54) - \log(0.1))}{286}} \right) > 0.9 \\
				\Rightarrow &err_D(h) \leq 0.03 + \sqrt{\frac{(\log(54) - \log(0.1))}{286}} \simeq 0.208149 \simeq 0.21
			\end{align*}
	\item	Using Theorem 3.4:
			\begin{align*}
				\Pr_{T ~ \mathcal{D}^m} &\left( \forall h \in \mathcal{H}: \text{if $h$ is consistent with $T$, then } err_D(h) \leq \epsilon \right) 1 - \delta \\
				\Pr_{T ~ \mathcal{D}^m} &\left( \forall h \in \mathcal{H}: \text{if $h$ is consistent with $T$, then } err_D(h) \leq 0.01 \right) 0.9 \\
				\Rightarrow &\epsilon = 0.01, \delta = 0.1 \\\\
				m &\geq \frac{1}{\epsilon} \ln \left( \frac{\abs{\mathcal{H}}}{\delta} \right) \\
				m &\geq \frac{1}{0.01} \ln \left( \frac{3^3}{0.1} \right) \\
				m &\geq 100 (\ln(27) - \ln(0.1)) \simeq = 559.84 \\
				\Rightarrow &m \geq 560
			\end{align*}
\end{enumerate}


\section*{Exercise 2}

\section*{Exercise 3}

\section*{Exercise 4}
See \refname{appendix} for code.
% \begin{enumerate}[label=(\alph*)]
% 	\item 
% \end{enumerate}
\lstinputlisting{code/exercise_04_output.txt}

\section*{Exercise 5}
\begin{enumerate}[label=(\alph*)]
	\item See \refname{appendix} for code.
        \lstinputlisting{code/exercise_05_output.txt}
    \item \(w_1^{(4)}\) and \(w_3^{(4)}\) are different because with \(\gamma = 0.5\) we put a certain weight on exploration.
        Therefore, even with the same reward, different actions can have different weights as \(\gamma\) is part of the weight updating calculation.
\end{enumerate}


\section*{Exercise 6}


\section*{Appendix}\label{appendix}
\subsection*{Code for Exercise 4}
\lstinputlisting[language=Python]{code/exercise_04.py}
\subsection*{Code for Exercise 5}
\lstinputlisting[language=Python]{code/exercise_05.py}


\end{document}